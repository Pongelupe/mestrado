\documentclass[xcolor=dvipsnames,table]{beamer}

\usepackage{latexsym}
\usepackage[utf8]{inputenc}
\usepackage[brazil]{babel}
\usepackage{amssymb}
\usepackage{amsmath}
\usepackage{stmaryrd}
\usepackage{fancybox}
\usepackage{datetime}
\usepackage[T1]{fontenc}
\usepackage{graphicx}
\usepackage{graphics}
\usepackage{url}
\usepackage{algorithmic}
\usepackage{algorithm}
\usepackage{acronym}
\usepackage{array}
\usepackage{multirow}
\usepackage{listings}
\usepackage{animate}

\usepackage{natbib}
\bibliographystyle{ACM-Reference-Format}
\citestyle{acmauthoryear}

\newtheorem{definicao}{Definio}
\newcommand{\tab}{\hspace*{2em}}

\mode<presentation>
{
        \definecolor{colortexto}{RGB}{0,0,0}

        \setbeamertemplate{background canvas}[vertical shading][ bottom=white!10,top=white!10]
        \setbeamercolor{normal text}{fg=colortexto} 

        \usetheme{Warsaw}
}

\title{PondiônsTracker: A framework based on GTFS-RT to identify delays and estimate arrivals dynamically in Public Transportation Network} 

\author{
        Pedro Pongelupe Lopes
}

\institute{Programa de Pós-Graduação em Informática}
\date{\textbf{December 04 2023} }

\logo{\includegraphics[width=1cm]{images/logoPuc.png}}

\begin{document}

\begin{frame}
        \titlepage
\end{frame}

\begin{frame}{Contents}%[allowframebreaks]{Sumário}
        \tableofcontents[currentsection, hideothersubsections]
        %\tableofcontents[currentsection, hideothersubsections]
\end{frame}


\section{Introduction}
\begin{frame}{Introduction}
        \begin{block}{Motivation}
                \begin{itemize}
                        \item Public Transportation Network
                        \item Huge centers 
                        \item GTFS and GTFS-RT specifications
                \end{itemize}
        \end{block}
                \begin{figure}[H]
                        \centering
                        \includegraphics[scale=0.15]{images/chuva-engarrafamento.jpg}
                \end{figure}
\end{frame}

\begin{frame}{Motivation}
        \begin{block}{GTFS-RT Matching Identifiers Issue}
                To work with GTFS-RT, it is \textbf{required} to track vehicles in real-time. But, in some cases, it is not easy to match the identifier between a real-time record and the GTFS static data. In Rome in 2016, this issue was reported by \cite{bigdata}. We still face this issue in Belo Horizonte in 2023.
        \end{block}
\end{frame}

\begin{frame}{Objectives}
        \begin{block}{Main Objective}
                \begin{itemize}
                        \item Proposing and validating PondiônsTracker
                \end{itemize}
        \end{block}
        \begin{block}{Specific Objectives}
                \begin{itemize}
                        \item Collecting data from the real-time API and combining with the GTFS
                        \item Understanding if Belo Horizonte's delays are spatial and temporal dependent 
                        \item Comparing the arrival times defined at the GTFS with the ones generated by \textit{PondiônsTracker}.
                \end{itemize}
        \end{block}
\end{frame}


\section{Theoretical Reference}
\begin{frame}{Theoretical Reference}
        \begin{block}{Main Ideas}
                \begin{itemize}
                        \item Smart Cities
                        \item Urban Computing
                        \item Human Mobility
                        \item Graphs and Complex Network
                \end{itemize}
        \end{block}
\end{frame}
\begin{frame}{Public Transportation Network as a Complex Network}
        \begin{figure}[H]
                \centering
                \includegraphics[scale=0.3]{images/final_graph.png}
                \caption{Delimiting Trips into Routes}
        \end{figure}
\end{frame}



\section{PondiônsTracker}
\subsection{Architecture and Entities}
\begin{frame}{PondiônsTracker}
        \begin{block}{Overview}
                \textit{PondiônsTracker}\footnote{Available at \url{https://github.com/Pongelupe/PondionsTracker/}} 
                is a framework to enrich
                GTFS data with real-time data. 
                The name \textit{PondiônsTracker} is a small gag from the sonority of the expression{ \em bus stop}
                when pronounced in Portuguese with the accent from Minas Gerais.
        \end{block}
\end{frame}
\begin{frame}{PondiônsTracker's Architecture}
        \begin{figure}[H]
                \centering
                \includegraphics[scale=0.3]{images/arq-pondionstracker.drawio.png}
                \caption{\textit{PondiônsTracker}'s architecture diagram}
        \end{figure}
\end{frame}
\subsection{Data Module}
\begin{frame}{Data Module Overview}
        \begin{figure}[H]
                \centering
                \includegraphics[width=\textwidth]{images/datamodule.png}
        \end{figure}
\end{frame}


\begin{frame}{Data Providers}
        \begin{figure}[H]
                \centering
                \includegraphics[scale=0.45]{images/mddatamodule.png}
                \caption{\textit{DataModule}'s maven dependency}
        \end{figure}
\end{frame}

\subsection{Integration Module}
\begin{frame}{Integration Module}
        \begin{figure}[H]
                \centering
                \includegraphics[scale=.205]{images/integrationModuleCD.png}
                \caption{Integration Module Class Diagram}
        \end{figure}
\end{frame}
\begin{frame}{Integration Driver}
        \begin{figure}[H]
                \centering
                \includegraphics[width = \textwidth]{images/integrationDriverAD.drawio.png}
                \caption{Integration Driver Activity Diagram}
        \end{figure}
\end{frame}
\begin{frame}{Integration Driver - 1st Step}
        \begin{figure}[H]
                \centering
                \includegraphics[width = \textwidth]{images/integrationDriverAD(1st_step).png}
                \caption{Integration Driver Activity Diagram}
        \end{figure}
\end{frame}
\begin{frame}{Has to generate the expected time?}
        \begin{figure}[H]
                \centering
                \includegraphics[width = \textwidth]{images/arrival_time_def.png}
                \caption{$arrival\_time$ definition from $stop\_times.txt$}
        \end{figure}
\end{frame}
\begin{frame}{Integration Driver - 2nd Step*}
        \begin{figure}[H]
                \centering
                \includegraphics[width = \textwidth]{images/integrationDriverAD(2nd_step).png}
                \caption{Integration Driver Activity Diagram}
        \end{figure}
\end{frame}


\begin{frame}{Integration Driver - 3rd Step}
        \begin{figure}[H]
                \centering
                \includegraphics[width = \textwidth]{images/integrationDriverAD(3rd_step).png}
                \caption{Integration Driver Activity Diagram}
        \end{figure}
\end{frame}

\begin{frame}{Integration Driver - 4th Step}
        \begin{figure}[H]
                \centering
                \includegraphics[width = \textwidth]{images/integrationDriverAD(4th_step).png}
                \caption{Integration Driver Activity Diagram}
        \end{figure}
\end{frame}

\begin{frame}{Integration Driver - 5th Step}
        \begin{figure}[H]
                \centering
                \includegraphics[width = \textwidth]{images/integrationDriverAD(5th_step).png}
                \caption{Integration Driver Activity Diagram}
        \end{figure}
\end{frame}

\begin{frame}{Integration Driver - 6th Step}
        \begin{figure}[H]
                \centering
                \includegraphics[width = \textwidth]{images/integrationDriverAD(6th_step).png}
                \caption{Integration Driver Activity Diagram}
        \end{figure}
\end{frame}
\begin{frame}{Integration Module}
        \begin{figure}[H]
                \centering
                \includegraphics[scale=0.45]{images/mdIntegrationModule.png}
                \caption{\textit{IntegrationModule}'s maven dependency}
        \end{figure}
\end{frame}

\subsection{PondiônsTracker-BH}
\begin{frame}{PondiônsTracker-BH}
        \begin{block}{Overview}
                \textit{PondiônsTracker-BH}\footnote{Available at \url{https://github.com/Pongelupe/PondionsTracker-BH}} is our \textit{PondiônsTracker}'s specialization to
                deal with Belo Horizonte's  Network. So, we have implemented our own \textit{Real-Time Data collector}, and we have overwritten a method from the \textit{RealTimeService} from the \textit{DataProviders}.
        \end{block}
\end{frame}
\begin{frame}{Belo Horizonte’s RealTimeService}
        \begin{block}{\textit{BHRealTimeService} - \textit{getIdsLineByRouteId}}
                Due to a  \textbf{one-to-many} relationship between the GTFS and the real-time data.
                \begin{itemize}
                        \item $Transfacil \rightarrow Traffic\ API$
                        \item $BHTrans \rightarrow GTFS$
                \end{itemize}
        \end{block}
\end{frame}

\section{Results}
\begin{frame}{Workload Overview}
        \begin{columns}
                \column{0.5\textwidth} 
                \begin{block}{Workload}
                        \begin{itemize}
                                \item Data collected for 11 days straight in August 2023 
                                \item 30 Gigabytes 
                        \end{itemize}
                \end{block}
                \column{0.5\textwidth}
                \begin{figure}[h]
                        \centering
                        \includegraphics[width=\textwidth]{images/workload.png}
                \end{figure}
        \end{columns}
\end{frame}
\subsection{Schedule Analysis}
\begin{frame}{Schedule Analysis}
        \begin{block}{\textbf{How much could our framework link the datasets?}}
                \textit{Schedule-Filled Percentage} = \textbf{Matched Trips} / \textbf{Scheduled Trips}
                \begin{itemize}
                        \item \textbf{Total}: 156,628 / 205,884 = 76.08\% 
                        \item \textbf{Weekdays}: 118,559 / 159,418 = 74.37\% 
                        \item \textbf{Saturdays}: 22,796 / 28,200 = 80.84\% 
                        \item \textbf{Sundays}: 15,273 / 18,266 = 83.61\% 
                \end{itemize}
        \end{block}
\end{frame}

\subsection{Delay Analysis}
\begin{frame}{Delay Analysis}
        \begin{block}{Delay Notation}
                When comparing a real-time to a expected time:
                \begin{itemize}
                        \item \textbf{Delay}: $\geqslant$ 1 minute after
                        \item \textbf{Ahead-of-Schedule}: $\geqslant$ 1 minute before
                        \item \textbf{On time}: $\leqslant$ 59 seconds after OR $\leqslant$ 59 seconds before
                \end{itemize}
        \end{block}
\end{frame}
\begin{frame}{Delay Analysis}
        \begin{columns}
                \column{0.5\textwidth}  		
                \begin{block}{Distribution of each status over the network}
                        \begin{itemize}
                                \item \textbf{Delay}: 89.8\% 
                                \item \textbf{Ahead-of-Schedule}: 6.9\% 
                                \item \textbf{On time}: 3.3\%
                        \end{itemize}
                \end{block}
                \column{0.5\textwidth}
                \centering
                \begin{block}{\textbf{Attention!}}
                        The predominance of $DELAYED$ in the Public Transportation Network \textbf{does not imply} that the network is not working
nor completely stopped!
                \end{block}
        \end{columns}
\end{frame}
\begin{frame}{Delay Analysis}
        \begin{figure}[H]
                \centering
                \includegraphics[width=\textwidth]{images/delays_distribution.png}
                \caption{$DELAY$s Distribution: Bus Stop and Trip}
        \end{figure}
\end{frame}
\begin{frame}{Delay Analysis}
        \begin{columns}
                \column{0.5\textwidth}  		
                \begin{figure}[H]
                        \centering
                        \caption{300 Most Delayed Stops}
                        \includegraphics[height=5cm, keepaspectratio]{images/mostDelayedStops.png}
                \end{figure}
                \column{0.5\textwidth}
                \begin{figure}[t]
                        \centering
                        \caption{Fragment of the 50 Most Delayed Stops }
                        \includegraphics[width=\linewidth]{images/10-50MostDelayedStops.png}
                \end{figure}
        \end{columns}
\end{frame}
\begin{frame}{Delay Analysis}
        \begin{block}{Network Constants} 
                \begin{enumerate}
                        \item \textit{Global Ahead Average}: $13.42$ minutes 
                        \item \textit{Global Delay Average}: $20.49$ minutes 
                \end{enumerate}
        \end{block}
        \begin{block}{Three Most Delayed Stops for Weekdays}
                \begin{enumerate}
                        \item $\#14793268$ - \textit{Avenida Dom Pedro II 1520} with $7,309$ delays 
                        \item $\#14791617$ - \textit{Avenida Amazonas 7309} with $7,009$ delays
                        \item $\#14790997$ - \textit{Avenida Dom Pedro II 1980} with $6,692$ delays 
                \end{enumerate}
        \end{block}
\end{frame}
\begin{frame}{Delay Analysis}
        \begin{block}{Stops $\#14793268$ and $\#14790997$} 
                The stops $\#14793268$ and $\#14790997$ are the first and third most delayed in the Public Transportation Network, respectively.
                Also, these stops are \textbf{462} meters from each other on the same avenue, \textit{Avenida Pedro II}, and share \textbf{2,590} common trips, so they are spatially related.
        \end{block}
        \begin{block}{\textit{Local Out-Of-Schedule Average}} 
                \begin{itemize}
                        \item $\#14793268$: $19.29$ minutes 
                        \item $\#14790997$: $19.68$ minutes 
                \end{itemize}
        \end{block}
\end{frame}
\begin{frame}{Delay Analysis}
        \begin{figure}[H]
                \centering
                \includegraphics[height=6.5cm,width=10cm]{images/stops2.png}
        \end{figure}
\end{frame}

\subsection{Comparison Between Generated and Real Data}
\begin{frame}{Comparison Between Generated and Real Data}
        \begin{block}{Overview} 
                The previous analysis was only possible because Belo Horizonte's GTFS defines the expected time for all bus stops on every trip. The \textit{Trip Expected Time Generator} generates the expected times when missing, so, we executed this component with Belo Horizonte's data and compared the expected times generated
with those defined at the GTFS.
        \end{block}
\end{frame}
\begin{frame}{Comparison Between Generated and Real Data}
        \begin{table}[h!]
                \centering
                \begin{tabular}{|c|c|c|r|r|}
                        \hline
                        \multicolumn{3}{|c|}{} &   GTFS  &  Generated  \\
                        \hline
                        \multirow{3}{*}{Weekday} & \multicolumn{2}{|c|}{$ON\_TIME$} & $3.3\%$ & $3.2\%$ \\\cline{2-5}
                                                 & \multicolumn{2}{|c|}{$AHEAD\_OF\_SCHEDULE$} & $6.9\%$ & $17.8\%$ \\\cline{2-5}
                                                 & \multicolumn{2}{|c|}{$DELAYED$} &$89.8\%$ & $79.0\%$  \\
                                                 \hline
                        \multirow{3}{*}{Saturday} & \multicolumn{2}{|c|}{$ON\_TIME$} & $3.9\%$ & $3.5\%$ \\\cline{2-5}
                                                  & \multicolumn{2}{|c|}{$AHEAD\_OF\_SCHEDULE$} & $6.5\%$ & $18.4\%$ \\\cline{2-5}
                                                  & \multicolumn{2}{|c|}{$DELAYED$} &$89.6\%$ & $78.1\%$  \\
                                                  \hline
                        \multirow{3}{*}{Sunday} & \multicolumn{2}{|c|}{$ON\_TIME$} & $4.4\%$ & $3.9\%$ \\\cline{2-5}
                                                & \multicolumn{2}{|c|}{$AHEAD\_OF\_SCHEDULE$} & $5.4\%$ & $18.5\%$ \\\cline{2-5}
                                                & \multicolumn{2}{|c|}{$DELAYED$} &$90.2\%$ & $77.6\%$  \\
                                                \hline
                \end{tabular}
        \end{table}
\end{frame}
\begin{frame}{Comparison Between Generated and Real Data}
        \begin{columns}
                \column{0.5\textwidth}  		
                \begin{block}{Global Averages} 
                        \begin{itemize}
                                \item \textit{Global Ahead Average} 
                                        \begin{itemize}
                                                \item GTFS: $13.42$ minutes 
                                                \item Generated: $38.57$ minutes 
                                                \item \textbf{Diff}: $25.15$ minutes 
                                        \end{itemize}
                                \item \textit{Global Delay Average} 
                                        \begin{itemize}
                                                \item GTFS: $20.49$ minutes 
                                                \item Generated: $24.75$ minutes 
                                                \item \textbf{Diff}: $4.26$ minutes 
                                        \end{itemize}
                        \end{itemize}
                \end{block}
                \column{0.5\textwidth}
                \begin{block}{\textit{Local Out-Of-Schedule Average}} 
                        \begin{itemize}
                                \item $\#14793268$
                                        \begin{itemize}
                                                \item GTFS: $19.29$ minutes 
                                                \item Generated: $15.54$ minutes 
                                                \item \textbf{Diff}: $3.75$ minutes 
                                        \end{itemize}
                                \item $\#14790997$ 
                                        \begin{itemize}
                                                \item GTFS: $19.68$ minutes 
                                                \item Generated: $14.68$ minutes 
                                                \item \textbf{Diff}: $5$ minutes 
                                        \end{itemize}
                        \end{itemize}
                \end{block}
        \end{columns}
\end{frame}



\section{Conclusion}
\begin{frame}{Conclusion}
        \begin{block}{Concluding Remarks}
                \begin{itemize}
                        \item Delays in Belo Horizonte follow a {\em log-normal} distribution
                        \item Analysis using data generated with the \textit{Trip Expected Time Generator}
                        \item \textit{PondiônsTracker} as a viable option when GTFS-RT is unavailable
                \end{itemize}

        \end{block}
        \begin{block}{Future Work}
                \begin{itemize}
                        \item Futher explore Belo Horizonte Public Transportation Network using deep learning for graphs approaches 
                        \item Reproduce Belo Horizonte's results with other cities
                \end{itemize}

        \end{block}
\end{frame}

\bibliography{ref}

\begin{frame}{Conclusion}
        Thanks!!
\end{frame}
\end{document}
