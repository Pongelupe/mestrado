% Abstract
\begin{abstract}
% Diminuir espaçamento entre título e texto
\vspace{-1cm}
% Texto do resumo, em inglês: sem paragrafo, justificado, com espaçamento 1,5 cm
\onehalfspacing
\noindent

A smart city Public Transportation Network provides mobility services to millions of people daily. 
Urban computing provides a
toolkit to handle acquisition, integration, and analysis, which translates to the improvement of the human mobility
of the citizens, mitigating and notifying delays, for instance.
Regarding the PTN, many methods rely on two specifications: \textit{General Transit Feed Specification} (GTFS) and \textit{General Transit Feed Specification Real-Time} (GTFS-RT). The first represents the static schedule information, and the second introduces real-time updates from trips, services, and vehicle positions. Despite the qualitative leap with the GTFS-RT specification, GTFS-RT
is not as well-adopted as the GTFS because of the non-existence of a matching identifier between the static and real-time data. In this context, we present \textit{PondiônsTracker}\footnote{Available at \url{https://github.com/Pongelupe/PondionsTracker/}} which is a {\em loose coupling} Java framework designed for enriching GTFS data with real-time data to enable delays analysis and to estimate arrivals. So, we present \textit{PondiônsTracker-BH}\footnote{Available at \url{https://github.com/Pongelupe/PondionsTracker-BH}} that is a \textit{PondiônsTracker}'s specialization created to
deal with Belo Horizonte's PTN particularities originating from Belo Horizonte's real-time \ac{API} 
which we collected every minute for eleven days straight in July and August 2023, summarizing over 246 million entries 
representing almost 30 Gigabytes.
\textit{PondiônsTracker-BH} presented a $76.08\%$ of the matched trips for the schedule during the observation period, 
then $156,628$ out of $205,884$ scheduled trips were identified in the real-time data. Analyzing these $156,628$ matched trips,
we show the delays focus and show that the delays in Belo Horizonte are spatial and temporal related
and are {\em log-normal} distributed. In other words, most of the delays in Belo Horizonte occur at a few bus stops, and these stops are \textit{physically} close to each other and share the same \textit{temporal} patterns.

% Espaçamento para as palavras-chave
\vspace*{.75cm}

% Palavras-chave: sem parágrafo, alinhado à esquerda
\noindent Keywords: Urban Computing, Human Mobility, Complex Networks, Public Transportation Network.
% Segunda linha de palavras-chave, com espaçamento.
%\indent\hspace{1.4cm} Keyword.

\end{abstract}
