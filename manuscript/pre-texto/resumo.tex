% Resumo
\begin{resumo}
% Diminuir espaçamento entre título e texto
\vspace{-1cm}

% Texto do resumo: sem paragrafo, justificado, com espaçamento 1,5 cm
\onehalfspacing

\noindent 
  O ano de 2020 vem sendo marcado pelo espalhamento em níveis globais de uma nova doença
  infecciosa, a COVID-19. Técnicas de computação urbana vem sendo empregadas para amenizar
  os danos sociais gerados pela pandemia, como utilização de rastreio de contato
  para medidas de isolamento social menos abruptas. Este projeto explorará evidências relacionadas aos 
  boletins epidemiológicos e dados de mobilidade para analisar a transmissão e disseminação
  da COVID-19 e propor medidas que minimizem os impactos da doença na sociedade. As análises
  buscam suavizar as medidas de isolamento social e planejamento do transporte urbano na cidade,
  assim propondo um método de rastreio de contatos.

% Espaçamento para as palavras-chave
\vspace*{.75cm}

% Palavras-chave: sem parágrafo, alinhado à esquerda
\noindent Palavras-chave: Computação Urbana, Mobilidade, Saúde, Redes Complexas
% Segunda linha de palavras-chave, com espaçamento.
%\indent\hspace{2cm}Palavra.

\end{resumo}