% Nome do capítulo
\chapter{Concluding Remarks}
% Label para referenciar
\label{cap5}

% Diminuir espaçamento entre título e texto
\vspace{-1.9cm}

In this master dissertation, we present \textit{PondiônsTracker}, a framework used to collect and
integrate GTFS and real-time data. Then, we validate our framework by specializing it to work with Belo Horizonte, \textit{PondiônsTracker-BH}. We collected data from the real-time API for eleven days straight, summarizing over 253 million entries in 2023 and combining it with Belo Horizonte's GTFS data to analyze 
its PTN. 

Our analysis compares the expected schedule with the actual schedule, which uncovers some gaps between
the schedule defined at the GTFS and the data collected at Belo Horizonte. First, there are a couple of routes, representing 385 trips the API has not reported at least one entry, and others 200 that have no matched trip, despite having reported entries. 
This scenario shows a lack of information from the real-time data provider. Also, this analysis points out
to the fact that the GTFS data provider under-scheduled trips, which were returned by the API.

Also, we shed light on some delay analysis using the expected times defined at the GTFS and the expected times
generated by the \textit{Trip Expected Time Generator}. The performance of the generated dataset shows that the
component is a viable option when the stop times are not defined. Belo Horizonte's data 
demonstrates that only four out of over 150 million trips were entirely on time compared to the schedule, and there are
many trips entirely out of schedule. This denotes how complex the PTN of a major city can get and points to the open
questions about delays currently being researched in the literature. 

Regarding delays in Belo Horizonte's PTN, we show that delays are the most predominant status concerning the schedule. 
They follow a {log-normal} distribution throughout the bus stops and might be both spatial and temporal related.
This scenario supplies substrate to guide the local programs and initiatives to address mobility issues because it
is the right of the citizens to transit around their city, and it directly affects thousands of people in a daily routine. 
Also, a concise PTN plays a significant role in approaching climate change questions because it could replace 
several individual vehicles with a few collective ones, for instance.

The analysis performed over Belo Horizonte was only possible because \textit{PondiônsTracker} enabled
the link between the two datasets because of the nonexistence of GTFS-RT. 
Developing, validating, and, mainly, sharing \textit{PondiônsTracker}
to as many cities as possible to analyze their data when the GTFS-RT is unavailable is our main
contribution in this master thesis. This is translated to \textit{PondiônsTracker} architecture, which
takes loose coupling as principal, and the \textit{DataProviders} and \textit{IntegrationModule}'s all components are available as Maven dependencies. 
Furthermore, the \textit{IntegrationModule} provides many components that are planned to be used as 
building blocks, so for each particular city, the different blocks can be adapted and combined. 
The \textit{IntegrationDriver} executes a default sequence of steps to unify the two datasets into a 
single SQL schema, which schema could have been created and populated by the shell script provided, $init.sh$.

As future work, we intend to explore Belo Horizonte's PTN further, then collect wider time intervals
and apply deep learning for graph techniques. This toolkit can explore the impact of different bus stops on each other, even if
they are apparently unrelated, such as two stops from different neighborhoods.
Also, \textit{PondiônsTracker} can be incorporated to geovisualization tools such Layerbase \cite{layerbase}. 
In addition, \textit{PondiônsTracker-BH} output, the base of all analysis, can be used as input for a deep graph network. For instance, state-of-the-art 
techniques such as Flock of Starlings can be exploited. 

Also, in future work, we intend to reproduce the results obtained in Belo Horizonte in other cities, not
restrained to Brazil. It is an opportunity to improve human mobility in smart cities without the GTFS-RT but with multiple providers, as described in \citeonline{delays_bigdata}. Finally, in future work, further exploring the PTN delays combining temporal and spatial dimensions because vehicles in 
the same pathway will get stuck{ \em together} in a traffic jam, and a slow pace of traffic during rush hour follows the same principle, for instance.
