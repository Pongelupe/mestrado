% Nome do capítulo
\chapter{Introduction}
% Label para referenciar
\label{cap1}

% Diminuir espaçamento entre título e texto
\vspace{-1.9cm}

% Texto do capítulo

A smart city \ac{PTN} provides mobility services to
millions of people daily. This is far from being a trivial task. Many major cities have hundreds of bus lines operating every day.
The complexity and importance of this network
have been a study object for a long time, and to create and 
manage the schedules, many approaches could use
manual or computational methods. 
The many computational 
methods work with the \ac{GTFS} \cite{GTFS}
and its real-time extension, 
\ac{GTFS-RT} \cite{GTFS-RT}
which are industry standard specifications for sharing schedules and
associated geographic information.

On the one hand, GTFS represents the static data of the PTN and its 
main entities are the trips, routes, stops, stop times, and fares. 
Thus, this specification has been promoting research in multiple fields, 
such as creating multimodal applications, 
ridesharing, and data visualization 
\cite{GTFSExemples}.
On the other hand, GTFS-RT introduced
a whole new dimension with real-time updates from trips,
services, and vehicle positions. With GTFS-RT,
many researchers could take a deep dive into 
new topics of question, such as measures of disparities in service provision, temporal
variability, the role of relative travel times and costs in mode choice
\cite{GTFS-RT_delays_2017, GTFS-RT_delays_2022, GTFS-RT_delays_2022-ADWIN}.

Both GTFS and GTFS-RT specifications are based on open data, which
the \ac{OKF} \cite{okf} defines as
"open data and content can be freely used, modified, 
and shared by anyone for any purpose." Some cities provide
their open data through {\em open data portals}, which aggregate
a wide range of
datasets, such as urban planning, health, and tourism \cite{opendata}.
These cities, which use their data to enable political efficiency and social and 
cultural development for their citizens, are considered smart cities \cite{smart_cities_SLR}.
Many smart cities follow OKF's mission "to create a free, fair, and open future, advancing open knowledge 
as a design principle beyond just data." These portals have been contributing
to the massive popularity of these specifications.

In order to work with the huge volume of data daily produced in smart cities, 
urban computing provides a toolkit to handle acquisition, integration, and 
analysis of the data from multiple sources \cite{urban_computing}.
Regarding the PTN, these tools aim to improve the human mobility
of the citizens by creating models based on individuals' movement patterns.
So, one common approach is to model the PTN as a complex network, in which 
the bus stops represent the nodes set, and the edges set is represented 
by the arcs connecting two nodes, in other words, the streets \cite{ferber2012}. 
This modeling also embraces PTN's additional information about the
sets previously mentioned, information as bus coordinates and delays.
The fact that the PTN is updated, and basically, every new data input
from hundreds of buses around a city
expresses the complexity of this network.

Although the qualitative leap with the GTFS-RT specification, GTFS-RT
is not as well-adopted as the GTFS. One may associate this condition with
the lack of real-time data, but many major cities supply both datasets.
\citeonline{delays_bigdata} describe this scenario for Rome back in 2016
and \citeonline{routableTimetableGTFS} identified and proposed a method to 
shorten this gap for Toronto in 2017. This issue occurs mainly due to the 
ownership of the data, transit agencies, commonly, provide the real-time data, and the government, the static data. 
This leads to the situation where there is no matching identifier 
between the two datasets.

Yet in 2023,
GTFS-RT is unavailable for many major cities due 
to the matching identifier issue. Then,
in this thesis, we propose \textit{PondiônsTracker}, a framework
to enrich GTFS data with real-time data enabling { \em collecting } and { \em integrating }. \textit{PondiônsTracker} is designed to work with as many cities as possible, so its components
are replaceable and have their behaviors defined in interfaces. Thus, we
introduce \textit{PondiônsTracker-BH}, which is a specialization to deal with
Belo Horizonte's PTN and to reproduce delay analysis because Belo Horizonte supplies the GTFS and real-time data 
despite not providing GTFS-RT.




\section{Objectives}
The main objective of this master's dissertation is 
proposing and validating PondiônsTracker, which is a framework to identify delays
and improve the estimated arrival task
in Public Transportation Networks in cities with buses, real-time data, and GTFS. So, 
reach the main objective, we use Belo Horizonte's data with the following specific objectives:
\begin{enumerate*}
    \item Collecting data from the real-time API and combining with the GTFS;
    \item Understanding if Belo Horizonte's delays are spatial and temporal dependent by analyzing delays among bus stops;
    \item Comparing the arrival times defined at the GTFS with the arrival times generated by \textit{PondiônsTracker-BH};
\end{enumerate*}

\section{Master Thesis Structure}
This master thesis is organized as follows. In Chapter 2, we present the 
theoretical reference, then we discuss the related work. Next, 
we describe the methodology used to build PondiônsTracker in Chapter 4.
In Chapter 5, we present the results using PondiônsTracker to Belo Horizonte.
Finally, we discuss our conclusions and future work.
