% Nome do capítulo
\chapter{Related work}
% Label para referenciar
\label{cap3}

% Diminuir espaçamento entre título e texto
\vspace{-1.9cm}

% Texto do capítulo

In this Chapter, we present related work identifying delays 
and strategies to mitigate them in public transportation networks.
Also, we approach the issue that GTFS-RT could be unavailable despite
having GTFS and real-time data and how the literature has dealt to
this issue.



\citeonline{delays_bigdata} use the combination of GTFS data and real-time data
to analyze delays in the public transportation network of Rome and Stockholm, 
These cities had divergent data available, resulting in differences in the analysis
used GTFS for the static data. For real-time data,
Rome's data were provided by the Mobility Control Center of Roma Servizi
della Mobilità, but there was no identifier to connect the real-time data
and the static data. So, the link between sets was inferred by comparing the
expected arrival time and the real arrival time within a threshold. The 
results show the average delay for a stop, the average delay for a trip, 
and the average delay for a trip at a stop.

Stockholm's real-time data provided by the Swedish Transportation Administration
and it was easy to link with the static data, unlikely Rome. Also, the real-time
data was composed of real-time updates on routes, disruptions, delays, and 
arrival-departure time of buses, trains, trains, and boats at each stop, 
so characterizing the GTFS-RT specification.
\citeonline{delays_bigdata} used the dataset to examine
three questions:
\begin{enumerate*}
    \item Are the delays in the public transport network spatially
dependent?
    \item What are the factors contributing to delays?
    \item What methods best suit the analysis of large
real-time data streams?
\end{enumerate*}
The first question approach was to calculate the mode and standard deviation
of the delays and apply a simple Moran's I test, which determines a
significant spatial auto-correlation.
The second was tacked using an OLS model in which the average delay is the dependent variable,
and maximum speed, road types, number of lines, and number of lanes are the
independent variables. Finally, regarding the third question, the authors point out 
that spatially aware, computationally intelligent, and machine-learning techniques
to work with this kind of dataset.

We replicated the delay analysis that \citeonline{delays_bigdata} had done in Rome, and we discussed the three questions analyzed with Stockholm's data using
Belo Horizonte data. Belo Horizonte's dataset resembles Rome's because no easy-matching identifier exists. 
Although, we address the issues
with a different approach but also incorporating the key ideas of the algorithm used in Rome, as further discussed in Chapter \ref{cap4}.


Many papers identify delays in public transportation networks using the GTFS-RT 
data, despite the differences in measurement and mitigation of delays, it is
a common approach to model the public transportation network using graph theory.
\citeonline{GTFS-RT_delays_2017} describe that a random delay along legal, physical, or social constraints are the factors that produce delays. To mitigate these factors,
the planners added to the schedule 
a safety margin delay, called schedule padding. 
Furthermore, the schedule padding
is given by the time required to operate on any given segment of the route for
each segment, the padding is the time difference between the fastest and the average time 
of the remaining entries.
A fixed value for schedule padding may heavily influence
the transportation experience by delaying routes without needing to. Then, the takeaway is that
the padding must be proportional to the random delay, which is caused by some unpredictable factors such as: 
traffic conditions, wrecked vehicles, or the number of red lights. 

We molded the public transportation networks using graph theory as well,
and we adapted \citeonline{GTFS-RT_delays_2017}'s concept of schedule padding to 
measure the 
delays and to estimate arrival-departure windows. Because the 
arrival-departure window of a bus to a bus stop depends on whether the bus is
delayed, or on time, or ahead of schedule, which expresses a negative,
neutral, positive padding to the expected arrival time, respectively.

%\citeonline{GTFS-RT_delays_2022}.


\citeonline{routableTimetableGTFS} point out that GTFS' analysis that researchers
have been dealing with could be better addressed using real-time data to enrich 
GTFS data. Because the GTFS is based on schedules, which are projections 
about the services rather than observations.
In other words, the research questions should take into account the events
that interfere with the real composition of the network, the randomness of 
a random complex network. Despite a brief mention of the GTFS-RT
specification, \citeonline{routableTimetableGTFS} reveal one common issue with it,
that is, the non-compatibility of the real-time data and the GTFS. So, they 
propose an algorithm based on the monitoring of vehicles in real-time as their
locations are updated to unify the two datasets. After the data collection,
the algorithm's following steps are: 
\begin{enumerate*}
    \item Delimiting trips and blocks;
    \item Spatial matching and positional error handling;
    \item Determining stop times;
    \item Constructing the retrospective GTFS package.
\end{enumerate*}
Finally, using the enriched GTFS, they demonstrate the usage in Toronto.

%threashold d %
The \citeonline{routableTimetableGTFS}'s algorithm is really interesting; we
approach the real-time data in a very similar outline. 
To delimit trips, we used the vehicle's current distance en route instead of 
the headsign. To do spatial matching and positional error handling, they used
Open Source Routing Machine's map-matching algorithms to match entries to 
bus stops, we opted for keeping a low-coupling architecture using the 
Postgis and GTFS. 
For determining stop times, we used the same approach of estimating time
by linear interpolation from the surrounding vehicle reports, but we propose 
a different heuristic to choose the reports to interpolate.
The architectural decisions we made and our justifications are further explored in
the following chapter, Chapter \ref{cap4}.

Given the state-of-the-art literature discussed in this chapter,
this dissertation contributes to GTFS and GTFS-RT because we present
a tool to link GTFS with real-time data in cities where GTFS-RT is unavailable, allowing analysis such as delay analysis. Then, we contribute to 
delay analysis due to the reproduction of questions using the proposed framework.
